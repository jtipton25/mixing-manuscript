\documentclass[]{elsarticle} %review=doublespace preprint=single 5p=2 column
%%% Begin My package additions %%%%%%%%%%%%%%%%%%%
\usepackage[hyphens]{url}

  \journal{Computers and Geoscience} % Sets Journal name


\usepackage{lineno} % add
  \linenumbers % turns line numbering on
\providecommand{\tightlist}{%
  \setlength{\itemsep}{0pt}\setlength{\parskip}{0pt}}

\usepackage{graphicx}
\usepackage{url}
\usepackage{amsmath}
\usepackage{float}
\usepackage{subcaption}
\captionsetup{compatibility=false}
\usepackage{setspace}
\usepackage{lineno}
\usepackage{textcomp}
\usepackage{gensymb}
%
\usepackage{xcolor} %% package for text color
%
\newcommand\numberthis{\addtocounter{equation}{1}\tag{\theequation}}
\usepackage{booktabs} % book-quality tables
%%%%%%%%%%%%%%%% end my additions to header

\usepackage[T1]{fontenc}
\usepackage{lmodern}
\usepackage{amssymb,amsmath}
\usepackage{ifxetex,ifluatex}
\usepackage{fixltx2e} % provides \textsubscript
% use upquote if available, for straight quotes in verbatim environments
\IfFileExists{upquote.sty}{\usepackage{upquote}}{}
\ifnum 0\ifxetex 1\fi\ifluatex 1\fi=0 % if pdftex
  \usepackage[utf8]{inputenc}
\else % if luatex or xelatex
  \usepackage{fontspec}
  \ifxetex
    \usepackage{xltxtra,xunicode}
  \fi
  \defaultfontfeatures{Mapping=tex-text,Scale=MatchLowercase}
  \newcommand{\euro}{€}
\fi
% use microtype if available
\IfFileExists{microtype.sty}{\usepackage{microtype}}{}
\bibliographystyle{elsarticle-harv}
\ifxetex
  \usepackage[setpagesize=false, % page size defined by xetex
              unicode=false, % unicode breaks when used with xetex
              xetex]{hyperref}
\else
  \usepackage[unicode=true]{hyperref}
\fi
\hypersetup{breaklinks=true,
            bookmarks=true,
            pdfauthor={},
            pdftitle={Short Paper},
            colorlinks=false,
            urlcolor=blue,
            linkcolor=magenta,
            pdfborder={0 0 0}}
\urlstyle{same}  % don't use monospace font for urls

\setcounter{secnumdepth}{5}
% Pandoc toggle for numbering sections (defaults to be off)


% Pandoc header



\begin{document}
\begin{frontmatter}

\title{A mechanistic approach to unmixing detrital geochronologic data using Bayesian nonparametric mixture models}

\author[a]{John R. Tipton%\fnmark[1]%\footnote{\samepage Contribution: conceptualization, model development, software, writing - original draft.}
}
\corref{1}
\fnmark[1]
\fntext[1]{\samepage Contribution: conceptualization, model development, software, and writing - original draft.}
\ead{jrtipton@uark.edu} 
\address[a]{University of Arkansas, Department of Mathematical Sciences, Fayetteville, AR, USA}
% \credit{conceptualization, model development, software, writing - original draft.}
% \fntext[fn1]{Contribution: conceptualization, model development, software, writing - original draft.}
  
\author[a]{Glenn R. Sharman
% \footnote{\samepage Contribution: conceptualization, model development, writing - review and editing.}
\fnmark[2]
}

\fntext[2]{Contribution: conceptualization, model development, and writing - review and editing.}

\author[b]{Samuel A. Johnstone 
% \footnote{\samepage Contribution: conceptualization, model development, writing - review and editing.}0
\fnmark[fn3]
}

\address[b]{U.S. Geological Survey, Geosciences and Environmental Change Science Center, Denver, USA}
\fntext[fn3]{\samepage Contribution: conceptualization and writing - review and editing.}


  
\begin{abstract}
Sedimentary deposits constitute the primary record of changing environmental conditions that have acted on Earth's surface over geologic time. Clastic sediment is eroded from source locations (parents) in sediment routing systems and deposited at sink locations (children). Both parents and children have characteristics that vary across many different dimensions, including grain size, chemical composition, and the geochronologic age of constituent detrital minerals. During transport, sediment from different parents is mixed together to form a child, which in turn may serve as the parent for other sediment further down system or later in time when buried sediment is exhumed. To the extent that parent sources produce sediment with distinguishable geochronologic ages, the distribution of detrital mineral ages observed in child sediments allows for investigation of the proportions of each parent in the child sediment which ultimately reflects properties of the sediment routing system, such as the relative sediment flux. To model the proportion of dates in a child sample that comes from each of the parent distributions, we use a Bayesian mixture of Dirichlet processes. This model allows for estimation of the mixing proportions with associated uncertainty while making minimal assumptions. We also present an extension to the model whereby we reconstruct unobserved parent distributions from multiple observed child distributions using mixtures of Dirichlet processes, accounting for uncertainty in both the number of parent distributions and the mixing proportions.
\end{abstract}

\end{frontmatter}



$body$

\bibliography{mixing.bib}   % name your BibTeX data base
% 
% % Non-BibTeX users please use
% \begin{thebibliography}{}
% %
% % and use \bibitem to create references. Consult the Instructions
% % for authors for reference list style.
% %
% \bibitem{RefJ}
% % Format for Journal Reference
% Author, Article title, Journal, Volume, page numbers (year)
% % Format for books
% \bibitem{RefB}
% Author, Book title, page numbers. Publisher, place (year)
% % etc
% \end{thebibliography}
% 
\end{document}
% % end of file template.tex

